\documentclass[]{article}
\usepackage{lmodern}
\usepackage{amssymb,amsmath}
\usepackage{ifxetex,ifluatex}
\usepackage{fixltx2e} % provides \textsubscript
\ifnum 0\ifxetex 1\fi\ifluatex 1\fi=0 % if pdftex
  \usepackage[T1]{fontenc}
  \usepackage[utf8]{inputenc}
\else % if luatex or xelatex
  \ifxetex
    \usepackage{mathspec}
  \else
    \usepackage{fontspec}
  \fi
  \defaultfontfeatures{Ligatures=TeX,Scale=MatchLowercase}
\fi
% use upquote if available, for straight quotes in verbatim environments
\IfFileExists{upquote.sty}{\usepackage{upquote}}{}
% use microtype if available
\IfFileExists{microtype.sty}{%
\usepackage{microtype}
\UseMicrotypeSet[protrusion]{basicmath} % disable protrusion for tt fonts
}{}
\usepackage[margin=1in]{geometry}
\usepackage{hyperref}
\hypersetup{unicode=true,
            pdftitle={Résumé des principaux résultats scientifiques du projet FRB BioSceneMada},
            pdfauthor={Ghislain Vieilledent},
            pdfborder={0 0 0},
            breaklinks=true}
\urlstyle{same}  % don't use monospace font for urls
\usepackage{natbib}
\bibliographystyle{plainnat}
\usepackage{graphicx,grffile}
\makeatletter
\def\maxwidth{\ifdim\Gin@nat@width>\linewidth\linewidth\else\Gin@nat@width\fi}
\def\maxheight{\ifdim\Gin@nat@height>\textheight\textheight\else\Gin@nat@height\fi}
\makeatother
% Scale images if necessary, so that they will not overflow the page
% margins by default, and it is still possible to overwrite the defaults
% using explicit options in \includegraphics[width, height, ...]{}
\setkeys{Gin}{width=\maxwidth,height=\maxheight,keepaspectratio}
\IfFileExists{parskip.sty}{%
\usepackage{parskip}
}{% else
\setlength{\parindent}{0pt}
\setlength{\parskip}{6pt plus 2pt minus 1pt}
}
\setlength{\emergencystretch}{3em}  % prevent overfull lines
\providecommand{\tightlist}{%
  \setlength{\itemsep}{0pt}\setlength{\parskip}{0pt}}
\setcounter{secnumdepth}{0}
% Redefines (sub)paragraphs to behave more like sections
\ifx\paragraph\undefined\else
\let\oldparagraph\paragraph
\renewcommand{\paragraph}[1]{\oldparagraph{#1}\mbox{}}
\fi
\ifx\subparagraph\undefined\else
\let\oldsubparagraph\subparagraph
\renewcommand{\subparagraph}[1]{\oldsubparagraph{#1}\mbox{}}
\fi

%%% Use protect on footnotes to avoid problems with footnotes in titles
\let\rmarkdownfootnote\footnote%
\def\footnote{\protect\rmarkdownfootnote}

%%% Change title format to be more compact
\usepackage{titling}

% Create subtitle command for use in maketitle
\newcommand{\subtitle}[1]{
  \posttitle{
    \begin{center}\large#1\end{center}
    }
}

\setlength{\droptitle}{-2em}
  \title{Résumé des principaux résultats scientifiques du projet FRB BioSceneMada}
  \pretitle{\vspace{\droptitle}\centering\huge}
  \posttitle{\par}
  \author{Ghislain Vieilledent}
  \preauthor{\centering\large\emph}
  \postauthor{\par}
  \predate{\centering\large\emph}
  \postdate{\par}
  \date{20/06/2017}

\hypersetup{colorlinks=true, citecolor=blue}

\begin{document}
\maketitle

\section{Rappel des objectifs du
projet:}\label{rappel-des-objectifs-du-projet}

L'objectif du projet BioSceneMada est de fournir des scénarios
d'évolution de la biodiversité sous l'impact conjoint du changement
climatique et de la biodiversité à Madagascar. Suite aux trois premières
années du projet concernant le volet scientifique, voici les principaux
résultats que nous avons obtenus. Ces résultats seront consolidés et
complétés pendant les deux dernières années du projet en parrallèle des
activités de renforcement de capacité des institutions à Madagascar. Ces
ateliers de renforcement de capacité assureront le transfert des
méthodes et outils scientifiques développés au cours des trois premières
années du projet afin de déterminer des stratégies de conservation
efficaces de la biodiversité.

\section{Principaux résultats
scientifiques:}\label{principaux-resultats-scientifiques}

\subsection{Carte de carbone forestier et vulnérabilité des forêts
tropicales au changement climatique à
Madagascar}\label{carte-de-carbone-forestier-et-vulnerabilite-des-forets-tropicales-au-changement-climatique-a-madagascar}

Dans le cadre du projet BioSceneMada, en utilisant les données
climatiques précédemment calculées (\url{https://madaclim.cirad.fr}) et
des données d'inventaires forestiers pour 1771 placettes réparties sur
l'ensemble de Madagascar, nous avons démontré qu'il existait un lien
fort entre climat et stocks de carbone forestiers. Ce lien est notamment
déterminé par les caractéristiques architecturales (hauteur notamment)
des espèces d'arbres présentes le long du gradient climatique à
Madagascar (climat --\textgreater{} assemblage d'espèces
--\textgreater{} stocks de carbone). Ainsi, les stocks de carbone sont
en moyenne beaucoup plus faibles en forêt épineuse (17 Mg.ha-1) qu'en
forêt humide (150 Mg.ha-1). Le modèle statistique intégrant la relation
climat-stock de carbone a permis de produire une carte précise des
stocks de carbone forestier à Madagascar à une résolution de 250 m.
Cette carte pourra être utilisée par les instances gouvernementales à
Madagascar ou les porteurs de projet REDD+ au niveau régional pour le
calcul des émissions de CO2 associées à la déforestation. Cette carte
ainsi que les données qui ont permis de l'obtenir sont disponibles sur
le site du projet BioSceneMada
(\url{https://bioscenemada.cirad.fr/carbonmaps}) ainsi que sur le
serveur de données Dryad (doi:
\href{http://doi.org/10.5061/dryad.9ph68}{10.5061/dryad.9ph68}).

Concernant les scénarios d'évolution de la biodiversité et des stocks de
carbone forestier, nous avons montré à l'aide de ce modèle que les
changements climatiques devraient induire des modifications fortes des
communautés forestières et en conséquence une diminution de -17\%
(7-24\%) des stocks de carbone forestier à Madagascar à l'horizon 2100
par rapport à 2010. Ces changements seront vraisemblablement plus forts
pour la forêt humide de l'est (notamment autour de la péninsule de
Masoala-Makira) que pour les forêts sèches et épineuses de l'ouest et du
sud. En comparaison, un taux de déforestation constant de 0.5\% par an
conduirait à une perte de carbone forestier de l'ordre de 29\% entre
2010 et 2100. L'impact potentiel des changements climatiques sur les
émissions de CO2 n'est donc pas à négliger. Ces résultats ont été
acceptés pour publication dans la revue Journal of Ecology avec la
référence ci-dessous:

\textbf{{Vieilledent G.}, O. Gardi, C. Grinand, C. Burren, M.
Andriamanjato, C. Camara, C. J. Gardner, L. Glass, A. Rasolohery, H.
Rakoto Ratsimba, V. Gond, J.-R. Rakotoarijaona}. 2016. Bioclimatic
envelope models predict a decrease in tropical forest carbon stocks with
climate change in Madagascar. \emph{Journal of Ecology}. \textbf{104}:
703-715. doi:
\href{http://dx.doi.org/10.1111/1365-2745.12548}{10.1111/1365-2745.12548}.

Cet article a constitué le choix de l'éditeur pour le numéro
\textbf{104}(3) de la revue Journal of Ecology:
\href{https://jecologyblog.wordpress.com/2016/05/06/editors-choice-1043/}{Editor's
Choice 104:3}.

Cet article a également été sélectionné par les éditeurs pour un numéro
spécial de la revue Journal of Ecology intitulé ``Plants in a changing
world: global change and plant ecology''. Publié le 24 April 2017.
\href{http://besjournals.onlinelibrary.wiley.com/hub/issue/10.1111/\%28ISSN\%291365-2745.globalchangevirtualissue}{Feuilleter
le numéro spécial}.

Les résultats de l'article ont été relayés dans plusieurs médias et
rapports d'activités: \href{https://t.co/pMXLUUrV0I}{The Conversation},
\href{/images/media/Figaro-16-02-2016.png}{Le Figaro},
\href{http://www.radioclassique.fr/player/progaction/initPlayer/podcast/3-minutes-pour-la-planete-2016-02-16-06-48-50.html}{Radio
Classique},
\href{http://www.lepoint.fr/environnement/le-rechauffement-climatique-risque-d-empecher-les-forets-tropicales-de-stocker-le-carbone-12-02-2016-2017587_1927.php\#xtor=RSS-221}{Le
Point},
\href{http://www.francetvinfo.fr/monde/environnement/le-rechauffement-climatique-risque-d-empecher-les-forets-tropicales-de-stocker-le-carbone_1312341.html\#xtor=AL-54-\%5Barticle\%5D}{FranceTV
info},
\href{http://www.emol.com/noticias/Tecnologia/2016/02/12/788109/Estudio-asegura-que-cambio-climatico-amenaza-la-absorcion-de-CO2-por-bosques-tropicales.html}{El
Mercurio}, \href{/images/media/MidiLibre-16-02-2016.png}{Midi-Libre},
\href{http://www.cirad.fr/en/news/all-news-items/press-releases/2016/climate-change-alters-the-co2-storage-capacity-of-tropical-forests}{Cirad},
\href{http://www.cirad.fr/content/download/11005/128917/version/2/file/RA2015_FR.pdf}{Cirad
activity report 2015}.

\subsection{2. Cartes historiques de la déforestation à Madagascar:
soixante ans d'étude de la déforestation et de la fragmentation
forestière.}\label{cartes-historiques-de-la-deforestation-a-madagascar-soixante-ans-detude-de-la-deforestation-et-de-la-fragmentation-forestiere.}

\subsection{\texorpdfstring{3. Logiciel Python `deforestprob' pour le
calcul de la probabilité spatiale de
déforestation}{3. Logiciel Python deforestprob pour le calcul de la probabilité spatiale de déforestation}}\label{logiciel-python-deforestprob-pour-le-calcul-de-la-probabilite-spatiale-de-deforestation}

\subsection{4. Modèles d'évolution de l'intensité de déforestation en
Afrique et à
Madagascar}\label{modeles-devolution-de-lintensite-de-deforestation-en-afrique-et-a-madagascar}

\subsection{4. Scénarios d'évolution de la couverture forestière à
Madagascar et cartes du couvert forestier
futur}\label{scenarios-devolution-de-la-couverture-forestiere-a-madagascar-et-cartes-du-couvert-forestier-futur}

\subsection{5. MadaClim: portail de données climatiques et
environnementales pour
Madagascar.}\label{madaclim-portail-de-donnees-climatiques-et-environnementales-pour-madagascar.}

Dans le cadre du projet BioSceneMada, nous avons développé le site
internet MadaClim (\url{https://madaclim.cirad.fr}). Ce site reprend
toutes les données climatiques actuelles fournies par WorldClim ainsi
que les prédictions climatiques issues des modèles du GIEC (groupe
d'experts intergouvernemental sur l'évolution du climat) et fournies par
le CGIAR CCAFS. Les données sont recompilées (reprojetées et
rééchantillonnées à 1km) et distribuées spécifiquement pour Madagascar.
Des variables bioclimatiques supplémentaires comme l'évapotranspiration
et le nombre de mois secs ont été calculées et ajoutées aux variables
déjà disponibles. En plus des données climatiques, des données
environnementales (sol, géologie, altitude, etc.) sont également
fournies. Ce site et ces données sont particulièrement utiles pour tous
les chercheurs, gestionnaires, membres d'ONG environnementales,
ministères voulant étudier les effets du changement climatique à
Madagascar. Elles peuvent être utilisées par exemple pour calculer les
anomalies climatiques prédites par les modèles du GIECC.

\subsection{6. Base de données de biodiversité à
Madagascar}\label{base-de-donnees-de-biodiversite-a-madagascar}

\subsection{7. Atlas de la biodiversité à Madagascar et de sa
vulnérabilité au changement
climatique}\label{atlas-de-la-biodiversite-a-madagascar-et-de-sa-vulnerabilite-au-changement-climatique}

\subsection{8. Cartes de biodiversité et des communautés d'espèces à
Madagascar}\label{cartes-de-biodiversite-et-des-communautes-despeces-a-madagascar}

\subsection{4. Communications:}\label{communications}

\begin{itemize}
\tightlist
\item
  Sites webs: du projet BioSceneMada et des outils MadaClim et
  deforestprob
\item
  Répertoires GitHub
\item
  Communication des résultats de recherche sur les réseaux sociaux
\item
  Conférences
\item
  Présentations
\item
  Rapports d'activités
\end{itemize}

\bibliography{bib/biblio.bib}


\end{document}
